\documentclass[a4paper]{article}

%% Language and font encodings
\usepackage[english]{babel}
\usepackage[utf8x]{inputenc}
%\usepackage[T1]{fontenc}

\usepackage{helvet}
\renewcommand{\familydefault}{\sfdefault}

%% Sets page size and margins
\usepackage[a4paper,top=3cm,bottom=2cm,left=3cm,right=3cm,marginparwidth=1.75cm]{geometry}

%% Useful packages
\usepackage{amsmath}
\usepackage{graphicx}
\usepackage[colorinlistoftodos]{todonotes}
\usepackage[colorlinks=true, allcolors=blue]{hyperref}


\usepackage{listings}
\lstset{basicstyle=\ttfamily,
  showstringspaces=false,
  commentstyle=\color{red},
  keywordstyle=\color{blue}
}



\title{Exercise 3: Training End-to-end driving networks}
\author{Michael Floßmann, Kshitij Sirohi, Hendrik Vloet}
\pagestyle{empty}
\begin{document}
\maketitle

\section{Introduction}
\subsection{Goals}

\begin{itemize}
	\item Implement and train the network \textit{command input} with augmented images
	\item Implement and train the network \textit{branched} with augmented images
	\item Implement and train \textit{branched} with non-augmented images
	\item Optional: try recurrent approaches
\end{itemize}

\subsection{Issues}

During the assignment we encountered several which delayed our development progress. Due to these problems we where not able to achieve the goals above. Below we will state a short list with the most problematic issues, attends to solve them and whether this worked or not. After that, a short list of results we can provide within this report will follow.

\begin{itemize}
	\item \textbf{Corrupted files}: While reading in the dataset, we encountered OSErrors. This was due to the corruption of one (and sometimes more than one) file. This file was basically empty and we had to write a "CheckCorruption" like routine to scan the dataset beforehand in order to avoid further crashes. Our routine works and removes corrupted files from the training set.
	\item \textbf{Disk Space}: Disk space was not sufficient to unpack the whole dataset and as consequence we where force to fetch the dataset from the university network which is slower than using the internal SSD (where access was restricted)
	\item \textbf{Runtime}: Training and Validation are quite time consuming, around 10h to 15h are normal for 5 epochs of training. We tried to speed it up by increasing the number of workers in the pytorch loader and accessing the validation sets in a more sequential way. This reduced the time consumption significantly.
	\clearpage
	\item \textbf{Nightly Reboots}: We wanted to train during the night, but due to the nightly reboots, the training was interrupted. We could do nothing regarding this problem. The only thing we could do was to save the model on a periodic basis and continue after the reboot with a new training run, but the old model.
	
\begin{lstlisting}[language=bash, caption={Nigthly Reboot Interruption}]

25153.85s - Train Epoch: 2 [344800/480640 (72%)]    Loss: 0.006571
25156.82s - Train Epoch: 2 [344900/480640 (72%)]    Loss: 0.006483
25159.51s - Train Epoch: 2 [345000/480640 (72%)]    Loss: 0.026645
25162.00s - Train Epoch: 2 [345100/480640 (72%)]    Loss: 0.019915
                                                                               
*** FINAL System shutdown message from root@login2 ***                       
System going down IMMEDIATELY                                                  
                                                                               
Nightly routine re-boot                                                        
                                                                               
25164.69s - Train Epoch: 2 [345200/480640 (72%)]    Loss: 0.006574
25167.05s - Train Epoch: 2 [345300/480640 (72%)]    Loss: 0.006452
25170.02s - Train Epoch: 2 [345400/480640 (72%)]    Loss: 0.027064
25173.58s - Train Epoch: 2 [345500/480640 (72%)]    Loss: 0.023583
Connection to login.informatik.uni-freiburg.de closed by remote host.
Connection to login.informatik.uni-freiburg.de closed.

\end{lstlisting}

\item \textbf{Retain Graph Error}: This error occurred two times in total. Both in the way like listed below. After the first time, we did as suggested by the program and set the \textbf{retain\_graph} attribute to true. But it did not help and the error did pop up again. This happened only on one of our private homestations, so we were basically forced to work on the pool computers.

\begin{lstlisting}[language=bash, caption={Nigthly Reboot Interruption}]


2251.09s - Train Epoch: 1 [120000/526080 (23%)]    Loss: 0.041661
2251.59s - Train Epoch: 1 [120050/526080 (23%)]    Loss: 0.039515
2252.20s - Train Epoch: 1 [120100/526080 (23%)]    Loss: 0.037241
2252.68s - Train Epoch: 1 [120150/526080 (23%)]    Loss: 0.035107
Traceback (most recent call last):
  File "./Branched.py", line 401, in <module>
    main()
  File "./Branched.py", line 359, in main
    loss.backward()
  File "/home/hive/pytorch_venv/lib/python3.5
  /site-packages/torch/tensor.py", line 93, in backward
    torch.autograd.backward(self, gradient, retain_graph,
     create_graph)
  File "/home/hive/pytorch_venv/lib/python3.5/site-packages/torch/
  autograd/__init__.py", line 89, in backward
    allow_unreachable=True)  # allow_unreachable flag
RuntimeError: Trying to backward through the graph a second time, 
but the buffers have already been freed. 
Specify retain_graph=True when calling backward the first time.

\end{lstlisting}

\item \textbf{Deus Ex Machina}: Somebody turns of a pool machine where we are training on.



\end{itemize}



\section{Architecture}
\subsection{Networks}



\subsection{Hyperparameters}

\subsection{Command Input}

\subsection{Branched}




\section{Results}


\end{document}