\documentclass[a4paper]{article}

%% Language and font encodings
\usepackage[english]{babel}
\usepackage[utf8x]{inputenc}
% \usepackage[T1]{fontenc}
\usepackage{float}
\usepackage{url}

\usepackage{helvet}
\renewcommand{\familydefault}{\sfdefault}

%% Sets page size and margins
\usepackage[a4paper,top=3cm,bottom=2cm,left=3cm,right=3cm,marginparwidth=1.75cm]{geometry}

%% Useful packages
\usepackage{amsmath}
\usepackage{graphicx}
\usepackage[colorinlistoftodos]{todonotes}
\usepackage[colorlinks=true, allcolors=blue]{hyperref}


\usepackage{listings}
\lstset{basicstyle=\ttfamily,
  showstringspaces=false,
  commentstyle=\color{blue},
  keywordstyle=\color{black}
}

\title{Exercise 4: Training End-to-end driving networks}
\author{Michael Floßmann, Kshitij Sirohi, Hendrik Vloet}
\pagestyle{empty}
\begin{document}
\maketitle

\section{Introduction}
\subsection{Goals}
The main objective of this assignment was to get to know the rAIScar hardware
and apply the things learned in the previous exercises to port the ML algorithm
to the platform. This included:
\begin{itemize}
\item Finish unfinished tasks from the previous exercise sheet
\item Get familiar with the rAIScar hardware to collect the training data
  (collecting and converting data for the training infrastructure)
\end{itemize} 

\section{Training process}
TODO

\section{Leftover tasks from Exercise 3}
\subsection{Issues}
There were several issues to overcome which weren't completely solved in time.
One big issue was that the meaning of the braking information was quite cryptic
and after asking the original paper authors of \cite{imitation}, it turned out
that the braking and acceleration data was supposed to be fused into one
variable. After this was done, the networks trained properly, as can be seen in
the loss plots (figures \ref{fig:augmented_command_loss}, \ref{fig:unaugmented_command_loss},
\ref{fig:augmented_branched_loss}, \ref{fig:unaugmented_branched_loss}).

The errors for the test set are shown in tables \ref{tab:error_command_aug} and \ref{tab:error_command_nonaug}.% TODO
\begin{table}[H]
  \centering
  \caption{Error values for the simulated command input net}
  \begin{tabular}{lccc|ccc}
    &\multicolumn{3}{c|}{Augmented} & \multicolumn{3}{c}{unaugmented} \\
    & MSE & (Median error)$^2$ & $R^2$-value & MSE & (Median error)$^2$ & $R^2$-value\\ \hline
    Steer & NaN & NaN & NaN & NaN & NaN & NaN \\
    Gas & NaN & NaN & NaN & NaN & NaN & NaN
  \end{tabular}
  \label{tab:error_command_aug}
\end{table}
\begin{table}[H]
  \centering
  \caption{Error values for the simulated branched net}
  \begin{tabular}{lccc|ccc}
    &\multicolumn{3}{c|}{Augmented} & \multicolumn{3}{c}{unaugmented} \\
    & MSE & (Median error)$^2$ & $R^2$-value & MSE & (Median error)$^2$ & $R^2$-value\\ \hline
    Steer & NaN & NaN & NaN & NaN & NaN & NaN \\
    Gas & NaN & NaN & NaN & NaN & NaN & NaN
  \end{tabular}
  \label{tab:error_command_nonaug}
\end{table}
\begin{figure}[!htbp]
  \centering
  \includegraphics[width=0.95\textwidth]{figures/sim_command_input_aug_lossplot}
  \label{fig:augmented_command_loss}
  \caption{Simulation command input lossplot}
\end{figure}
\begin{figure}[!htbp]
  \centering
  \includegraphics[width=0.95\textwidth]{figures/sim_command_input_nonaug_lossplot}
  \label{fig:unaugmented_command_loss}
  \caption{Simulation command input, unaugmented lossplot}
\end{figure}
\begin{figure}[!htbp]
  \centering
  \includegraphics[width=0.95\textwidth]{figures/sim_branched_aug_lossplot}
  \label{fig:augmented_branched_loss}
  \caption{Simulation branched lossplot}
\end{figure}
\begin{figure}[!htbp]
  \centering
  \includegraphics[width=0.95\textwidth]{figures/sim_branched_nonaug_lossplot}
  \label{fig:unaugmented_branched_loss}
  \caption{Simulation branched, unaugmented lossplot}
\end{figure}


\newpage{}
\section{Results for the Klinikum run}
For our rAIScar dataset, we went to the klinikum Freiburg, because the roads
there are narrow enough and there aren't too many bystanders which would disturb
the network.

For training, we wrote a python script, converting the rosbag files from the run
into h5 datasets in the form of the simulation.
\begin{figure}[!htbp]
  \centering
  \includegraphics[width=0.95\textwidth]{figures/klinikum_command_input_aug_lossplot}
  \label{fig:klinikum_augmented_command_loss}
  \caption{Klinikum rAIScar command input lossplot}
\end{figure}
\begin{figure}[!htbp]
  \centering
  \includegraphics[width=0.95\textwidth]{figures/klinikum_command_input_nonaug_lossplot}
  \label{fig:klinikum_unaugmented_command_loss}
  \caption{Klinikum rAIScar command input, unaugmented lossplot}
\end{figure}
\begin{figure}[H]
\centering
\includegraphics[width=0.95\textwidth]{figures/raiscar_branched_non_aug_klinikum_lossplot}
\label{fig:klinikum_unaugmented_branched_loss}
\caption{Klinikum rAIScar branched, unaugmented lossplot}
\end{figure}
\begin{figure}[H]
	\centering
	\includegraphics[width=0.95\textwidth]{figures/raiscar_branched_aug_klinikum_lossplot}
	\label{fig:klinikum_augmented_branched_loss}
	\caption{Klinikum rAIScar branched, augmented lossplot}
\end{figure}
\begin{thebibliography}{9}
\bibitem{imitation}
Codevilla, Felipe and Müller, Matthias and López, Antonio and Koltun, Vladlen
and Dosovitskiy, Alexey.
\textit{End-to-end Driving via Conditional Imitation Learning.}
ICRA 2018
\end{thebibliography}
\end{document}